\section{ctw.cpp}
\begin{verbatim}
/********************************************
 --------------------------------------------
ctw.cpp

      main routain for binary ctw coder



*********************************************/

#include <stdlib.h>
#include <time.h>
#include <fcntl.h>
#include <io.h>
#include "ctw_bcoder_incr.h"
#include "bitio_file.h"
#include "KrTr_esti.h"
#include "errormac.h"

#if !defined MAX
#define MAX(A,B) ((A)>(B)?(A):(B))
#endif

long depth=0, algo=0, x=0, source_length=-1,
  // depth: depth of context-tree if a 0
  // algo:type of algorithm
  // x:decompress(1) or compress(0)
  // source_length:source length
  no_out=0, max_depth=-1, max_nodes=-1, eta=1,
  // no_out: if 1, not output.
  // max_depth:maximum depth if a 1
  // max_nodes:maximum number of nodes if a 1
  // eta:condition;n>=eta
  verbose=1;
char *ofile_name=NULL, *ifile_name=NULL;
  // output file name and input file name

/***************  show usage  ***************/
void usage()
{
  cerr <<
    "Usage: ctw [-h][-?][-D depth][-MD depth]"
    "[-MN num][-eta num][-i size][-a n][-c][-s]"
    "[-x][-T trees] -o file1 file2\n"
    "  -h,-? : show this.\n"
    "  -D depth : depth of context tree."
        " default 0.\n"
    "  -MD depth : maximum depth if a 1."
        " default no limitation.\n"
    "  -MN num : maximum number of nodes"
      " if a 1. default no limitation.\n"
    "  -eta num : condition for incremental>=1;"
      " n>=eta."
    "default 1.\n"
    "  -i size : input data size (bits)."
        " default until eof.\n"
    "  -a 0 : use fixed-depth context set"
        "(default).\n"
    "  -a 1 : use incremental CTW.\n"
    "  -c : no output.\n"
    "  -s : silent mode.\n"
    "  -x : decompress. default compress.\n"
      " default 1.\n"
    "  -o file1 : output file.\n"
    "  file2 : input file.\n";
}

#define load_int(V) \
  egf( ++i >= argc );V = atol(argv[i]);break

/***********  parameter analysis  ***********/
int analyse_parameter( int argc, char *argv[] )
{
  int i;
  egf( argc == 1 );
  for( i=1; i < argc; i++ )
    if( argv[i][0]=='-' || argv[i][0]=='/' ){
      switch( argv[i][1] ){
        case 'a': load_int(algo);
        case 'D': load_int(depth);
        case 'e':
          if(argv[i][2]=='t'&&argv[i][3]=='a')
            load_int(eta);
          break;
        case 'M':
          switch( argv[i][2] ){
            case 'D': load_int(max_depth);
            case 'N': load_int(max_nodes);
            default: goto f;
          }
          break;
        case 'c': no_out=1; break;
        case 'i': load_int(source_length);
        case 'x': x = 1; break;
        case 's': verbose = 0; break;
        case 'o':
          egf( ++i >= argc || ofile_name );
          ofile_name = argv[i];
          break;
        case 'h': case '?': default: // usage
          goto f;
      }
    } else { // input-file name
      egf( ifile_name );
      ifile_name = argv[i];
    }
  return 0;
f:
  return -1;
}

/***** prepare bit-input for data output ****/
bit_input *prepare_input()
{
  bit_input_file *p_bif = NULL;

  e0gf( ifile_name );
  e0gfn( p_bif = new bit_input_file(), 1);

  if( p_bif-> init(ifile_name,x)<0 ){
    cerr<<"Can't open "<<ifile_name<<".\n";
    goto f;
  }
  if( !x && source_length < 0 )
    source_length = p_bif-> get_length();
  return p_bif;
f1:
  cerr << "memory error.\n";
f:
  delete p_bif;
  return NULL;
}

/**** prepare bit-output for data output ****/
bit_output *prepare_output()
{
  bit_output_file *p_bof = NULL;

  e0gf( ofile_name);
  e0gfn( p_bof = new bit_output_file(), 1);

  if( p_bof-> init(ofile_name,!x)<0 ){
    cerr<<"Can't open "<<ofile_name<<".\n";
    goto f;
  }
  return p_bof;
f1:
  cerr << "memory error.\n";
f:
  delete p_bof;
  return NULL;
}

/************  prepare ctw coder  ***********/
ctw_bcoder *
  prepare_ctw_bcoder
    (bit_input *p_bi,bit_output *p_bo)
{
  ctw_bcoder *p_ctw = NULL;
  ctw_bcoder_incr *p_ctwi = NULL;

  switch( algo ){
    case 0: // fixed-depth ctw
      p_ctw = new ctw_bcoder();
      e0gfn(p_ctw,1);
      emgfn( p_ctw->
        init(depth,
          KrTr_bupdate,p_bi,p_bo,x), 2 );
      return p_ctw;
    case 1: // incremental ctw
      p_ctwi = new ctw_bcoder_incr();
      e0gfn(p_ctwi,1);
      emgfn( p_ctwi->
        init(
          KrTr_bupdate,p_bi,p_bo,x,
            eta,max_depth,max_nodes), 2 );
      return p_ctwi;
    default:
      usage();
      goto f;
  }
f1:
  cerr << "memory error.\n"; goto f;
f2:
  cerr << "Can't build the coder.\n";
f:
  delete p_ctw; delete p_ctwi;
  return NULL;
}

/********************************************/
/*               main routain               */
/********************************************/
int main( int argc, char *argv[] )
{
  bit_input *p_bi = NULL;
  bit_output *p_bo = NULL;
  ctw_bcoder *p_ctw = NULL;
  time_t t;

  emgfn( analyse_parameter(argc,argv), 1);
  e0gf( p_bi = prepare_input() );
  if( !no_out ) e0gf( p_bo = prepare_output() );
  e0gf( p_ctw = prepare_ctw_bcoder(p_bi,p_bo) );
  if( !verbose ) p_ctw->verbose = 0;

  /** encode or decode **/
  t=time(NULL);
  if( !x ) p_ctw-> encode(source_length);
  else     p_ctw-> decode();
  if( verbose )
    cerr << "time: " << time(NULL)-t
         << "[sec]\n";
    cerr.flush();

  delete p_bi; delete p_bo; delete p_ctw;
  return 0;
f1:
  usage();
f:
  delete p_bi; delete p_bo; delete p_ctw;
  return -1;
}
\end{verbatim}
