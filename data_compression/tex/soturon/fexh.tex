\section{float\_ex.h}
\begin{verbatim}
/********************************************
 --------------------------------------------
float_ex.h

     double decimal, 32bits exponent float

  This class can be used
    as a substitution of float or double.
  The four basic operations with double
      are defined.
  To get double type value, use Double().
*********************************************/

#if !defined __float_ex_h__
#define __float_ex_h__

/********************************************/
/*  double decimal, 32-bit exponent float   */
/********************************************/
class float_ex
{
public:

  /* the value is 0.decimal * 2^exp */
  double decimal;
  int exp;

  float_ex( double i=0 );
  float_ex( double decmali, int expi );
  float_ex( const float_ex &one );

  float_ex operator-() const;
  float_ex &operator=( const float_ex &one );
  float_ex &operator=( double one );
  double Double() const;
  operator int() const;
  float_ex operator+(const float_ex &one) const;
  float_ex &operator+=( const float_ex &one );
  float_ex operator-(const float_ex &one) const;
  float_ex &operator-=( const float_ex &one );
  float_ex operator*(const float_ex &one) const;
  float_ex operator*( const double one ) const;
  float_ex &operator*=( const float_ex &one );
  float_ex operator/(const float_ex &one) const;
  float_ex operator/( const double one ) const;
  float_ex &operator/=( const float_ex &one );

  /** inner opearation **/

  void add(
    double &d, int &e,
    double d1, int e1,
    double d2, int e2
  ) const;
  void mul(
    double &d, int &e,
    double d1, int e1,
    double d2, int e2
  ) const;
  void div(
    double &d, int &e,
    double d1, int e1,
    double d2, int e2
  ) const;

};

#endif
\end{verbatim}
