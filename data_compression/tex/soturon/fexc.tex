\section{float\_ex.cpp}
\begin{verbatim}
/********************************************
 --------------------------------------------
float_ex.cpp

     double decimal, 32bits exponent float



*********************************************/

#include <float.h>
#include <math.h>
#include "float_ex.h"

#define sgn(x) ((x)? ((x)>0?1:-1) :0)

#define normalize(d,e)    \
if( d != 0 ){             \
  while( fabs(d) >= 1 ){  \
    d /= 2;               \
    e++;                  \
  }                       \
  while( fabs(d) < 0.50 ){\
    d *= 2;               \
    e--;                  \
  }                       \
} else

/***************  constructor  **************/
float_ex :: float_ex( double i )
{
  *this = i;
}
float_ex :: float_ex(double decimali,int expi)
{
  decimal = decimali;
  exp = expi;
}
float_ex :: float_ex( const float_ex &one )
{
  decimal = one.decimal;
  exp = one.exp;
}

/*****************  operator  ***************/
float_ex float_ex :: operator-() const
{
  return float_ex( -decimal, exp );
}
float_ex &float_ex ::
operator=( const float_ex &one )
{
  decimal = one.decimal;
  exp = one.exp;
  return *this;
}
float_ex &float_ex :: operator=( double one )
{
  decimal = frexp( one, &exp );
  return *this;
}
double float_ex :: Double() const
{
  return ldexp( decimal, exp );
}
float_ex :: operator int() const
{
  return (int)((*this).Double());
}
float_ex float_ex ::
operator+( const float_ex &one ) const
{
  double d;
  int e;
  add( d,           e,
       decimal,     exp,
       one.decimal, one.exp );
  return float_ex( d, e );
}
float_ex &float_ex ::
operator+=( const float_ex &one )
{
  add( decimal,     exp,
       decimal,     exp,
       one.decimal, one.exp );
  return *this;
}
float_ex float_ex ::
operator-( const float_ex &one ) const
{
  double d;
  int e;
  add( d,           e,
       decimal,     exp,
      -one.decimal, one.exp );
  return float_ex( d, e );
}
float_ex &float_ex ::
operator-=( const float_ex &one )
{
  add( decimal,     exp,
       decimal,     exp,
      -one.decimal, one.exp );
  return *this;
}
float_ex float_ex ::
operator*( const float_ex &one ) const
{
  double d;
  int e;
  mul( d,           e,
       decimal,     exp,
       one.decimal, one.exp );
  return float_ex( d, e );
}
float_ex float_ex ::
operator*( const double one ) const
{
  double d, d2;
  int e, e2;
  d2 = frexp( one, &e2 );
  mul( d,       e,
       decimal, exp,
       d2,      e2 );
  return float_ex( d, e );
}
float_ex &float_ex ::
operator*=( const float_ex &one )
{
  mul( decimal,     exp,
       decimal,     exp,
       one.decimal, one.exp );
  return *this;
}
float_ex float_ex ::
operator/( const float_ex &one ) const
{
  double d;
  int e;
  div( d,           e,
       decimal,     exp,
       one.decimal, one.exp );
  return float_ex( d, e );
}
float_ex float_ex ::
operator/( const double one ) const
{
  double d, d2;
  int e, e2;
  d2 = frexp( one, &e2 );
  div( d,       e,
       decimal, exp,
       d2,      e2 );
  return float_ex( d, e );
}
float_ex &float_ex ::
operator/=( const float_ex &one )
{
  div( decimal,     exp,
       decimal,     exp,
       one.decimal, one.exp );
  return *this;
}

/*************  inner operation  ************/
void float_ex ::
  add(
    double &d, int &e,
    double d1, int e1,
    double d2, int e2
  ) const
{
  double edif;

  if( d1 == 0 ){
    d = d2; e = e2;
    return;
  } else if( d2 == 0 ){
    d = d1; e = e1;
    return;
  } else
    if( e1 >= e2 ){
      edif = pow(2,e2-e1);
      d = d1 + d2 * edif;
      e = e1;
      normalize(d,e);
      return;
    } else {
      edif = pow(2,e1-e2);
      d = d2 + d1 * edif;
      e = e2;
      normalize(d,e);
      return;
    }
}

void float_ex ::
  mul(
    double &d, int &e,
    double d1, int e1,
    double d2, int e2
  ) const
{
  if( d1 == 0 || d2 == 0 ){
    d = 0; e = 0;
    return;
  }
  d = d1 * d2;
  e = e1 + e2;
  normalize(d,e);
  return;
}

void float_ex ::
  div(
    double &d, int &e,
    double d1, int e1,
    double d2, int e2
  ) const
{
  if( d1 == 0 ){
    d = 0;
    e = 0;
  } else {
    d = d1 / d2;
    e = e1 - e2;
    normalize( d, e );
  }
  return;
}
\end{verbatim}
